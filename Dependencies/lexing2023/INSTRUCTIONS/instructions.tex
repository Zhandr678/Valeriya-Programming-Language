
\documentclass{article}
\title{Using the Tokenizing Tools of Maphoon}
\author{Dina Muktubayeva and Hans de Nivelle}

\begin{document}

\maketitle
\begin{abstract}
   We describe how the tokenizing tools of MaphTT are
   used. 
\end{abstract}

\section{Quick Start for the Impatient}

Get the files from directory \verb+lexing+.
Create a main file that contains \\
\verb+main( int argc, char* argv[] ) { }.+
You can also use the existing file \verb+tester.cpp+. 
Make sure that the file contains the following includes: 
\begin{verbatim}
#include "algorithms.h"
#include "minimization.h"
#include "generators.h"
#include "deterministic.h" \end{verbatim}
Make sure that your compiler compiles it. 

\noindent
Now suppose we want to obtain the automaton for
regular expression 
\[  (\, a|b \, )^{*} \, | \, a^{*}. \]
Write: 
\begin{verbatim}
   using namespace lexing; 

   acceptor<char> reg = ( just('a') | just('b') ).star( ) |
                          just('a'). star( );

   std::cout << reg << "\n"; \end{verbatim}
This prints an acceptor that accepts the regular expression. 
As next step, we want to make this automaton deterministic.
Because MaphTT has no operations on 
\verb+acceptor+, except for constructing them, 
we need to obtain a classifier first.
A classifier has two template parameters, the alphabet
and the type that we are using for the token classes.
If one cares about efficiency, the token class
should be \verb+int+ or a dedicated \verb+enum+ type.
We don't care about efficiency, so we just use
\verb+std::string+. When a classifier is constructed,
one needs to provide the default classification
that will be used when classification fails.

\begin{verbatim}
   classifier< char, std::string > cl( "#error" );
   cl. insert( reg, "#ourlang" ); 
   std::cout << cl << "\n"; \end{verbatim}
When the classifier is printed, it only prints the
classifications of states that do not classify as
\verb+#error+. 
We see a lot of $ \epsilon $-transitions, starting
from \verb+Q0+, so we write
\begin{verbatim}
   cl = make_deterministic( cl ); 
   std::cout << cl << "\n"; \end{verbatim}
Unfortunately, this results in a \verb+logic_error+, 
with an additional message that \verb+empty word classifies as #ourlang+.
This is caused by the fact that the original
expression accepts the empty word, which would 
make it impossible to construct a usable tokenizer. 
The empty word
would be a valid token, and the tokenizer
would keep on returning this token forever. 
The problem can be solved by replacing 
\verb+.star( )+ by \verb+.plus( )+ or by adding
a terminator. We will add (;) as terminator.
Add the line 
\begin{verbatim}
   reg *= just( ';' );  \end{verbatim}
Now the code runs and constructs an automaton with
5 states. This is ridiculously many of course. 
We can add: 
\begin{verbatim}
   std::cout << "minimization\n";
   cl = minimize( cl ); 
   std::cout << cl << "\n"; \end{verbatim}
This results in a classifier with two states. 
The final classifier is readable. 
State \verb+Q0+ loops on $a,b$ and jumps to 
\verb+Q1+ when a (;) is encountered.
 
\section{Creating a Prototype} 

\noindent
Assume we want to create a calculator, whose tokens are
\begin{itemize}
\item
   Identifiers. 
\item
   Floating point numbers. 
\item
   A few operators: \  \verb|+, -, *, /, %|.
\item
   Opening and closing parentheses: \verb+( )+.
\item
   Semicolon to end the input: (\verb+;+). 
\end{itemize} 

Since we are going to care about efficiency eventually,
we create an enumeration type:
\begin{verbatim}
enum tclass
{
   tc_float = 1,
   tc_ident = 2,
   tc_add = 10, tc_sub = 11, tc_mul = 12, tc_div = 13, tc_mod = 14,
   tc_lpar = 20, tc_rpar = 21,
   tc_semicolon = 30
}; \end{verbatim}

\noindent
When creating a tokenizer, we recommend that one 
creates the classifier in a function. That 
will be easier for the later steps. 
So, we define
\begin{verbatim}
lexing::classifier< char, tclass > buildclassifier( )
{
   using namespace lexing;

   classifier< char, tclass > cl( tc_error );

   cl = make_deterministic(cl);
   cl = minimize(cl); 
   return cl;
} \end{verbatim}
Adding tokens is straightforward:
\begin{verbatim}
   auto idfirst = range('a', 'z') | range('A', 'Z');
   auto idnext = range('a', 'z') | range('A', 'Z') |
                 just('_') | range('0', '9');


   auto number = (just('0') |
               (range('1', '9') * range('0', '9').star()));

   cl. insert( idfirst * idnext.star(), tc_ident );

   auto exp = ( just( 'e' ) | just( 'E' )) *
                 ( just( '-' ) | just( '+' )). optional( ) *
                 ( range( '0', '9' ). optional( ) *
                   range( '0', '9' ). optional( ) *
                   range( '0', '9' ). optional( ));

   auto floatconst = number *
               ( just('.') * ( range('0', '9').plus()) ). optional( ) *
                 exp. optional( );

   cl. insert( floatconst, tc_float );

   cl. insert( just( '+' ), tc_add );
   cl. insert( just( '-' ), tc_sub );
   cl. insert( just( '*' ), tc_mul );
   cl. insert( just( '/' ), tc_div );
   cl. insert( just( '%' ), tc_mod );

   cl. insert( just( '(' ), tc_lpar );
   cl. insert( just( ')' ), tc_rpar );

   cl. insert( just( ';' ), tc_semicolon ); \end{verbatim}

   \noindent
   Token classes are printed as integers, which 
   is a bit unpleasant. 
   This problem can be solved by defining
   \verb+operator <<+ on \verb+tclass+.
   The parser generator\footnote{not ready for release} automatically creates 
   \verb+enum+ classes from the grammar with print functions.

   Currently, the classifier will reject whitespace,
   so we add \verb+tc_whitespace = 40+ to \verb+tclass+ and add
   the following code to \verb+buildclassifier+:
   \begin{verbatim}
   cl. insert( just( ' ' ) | just( '\t' ), tc_whitespace );

   auto linecomment = word("//") *
                    ( every<char>(). without('\n') ).star() * just('\n');
   cl. insert( linecomment, tc_whitespace );

   auto blockcomment = word("/*") *
     ( every<char>().without('*') |
       ( just('*').plus() * every<char>().without('/').without('*')).star()
     ).star() * just('*').plus( ) * just('/');

   cl. insert( blockcomment, tc_whitespace );  \end{verbatim}
   Now the classifier classifies whitespace and both types
   of $ C^{++} $-comments. 

   \noindent
   In addition to the constructors shown in the examples, one can also use
   \verb+word( )+, which can be called with a 
   string constant. 
\section{Using a Classifier}

In the previous section we constructed a classifier.
MaphTT does not create complete tokenizers, it only helps
with classifying. 
In particular, it doesn't help with computing token attributes.
We think it is no problem because most tokens have no attribute,
and computing attributes is easy when the token has been classified.

In order to classify input, one needs a class with at least the following 
methods:
\begin{itemize}
\item
   \verb+bool has(n)+. Make sure that there are at least
   $ n $ characters in the buffer. Return \verb+true+ if this 
   is guaranteed. 
\item
   \verb+C peek(i)+. Gets the $ i $-th character.
   \verb+has(n)+ with $ n > i $ must have been called before, 
   and must have succeeded. 
\item
   \verb+commit(i)+. Commit to $ i $ characters.
   This means that $ i $ characters will be removed from
   the buffer. 
   \verb+has(n)+ with $ n \geq i $ must have been called before, 
   and must have succeeded. 
\end{itemize}
We provide two classes that provide these methods: 
\verb+filereader+ and \verb+stringreader+.

\begin{itemize}
\item
   A filereader can be obtained as follows:
\begin{verbatim}
   filereader inp( &std::cin, "stdin" ); \end{verbatim}
or
\begin{verbatim}
   std::string name = "filename"; 
   std::ifstream inputfile( name ); 
   filereader inp( &inputfile, name ); \end{verbatim}
\item
   A stringreader can be obtained as follows:
\begin{verbatim}
   stringreader inp( "3.14 / pi + 10" ); \end{verbatim}
or
\begin{verbatim}
   std::string s = "input400 /*xxxx*/ ;";
   stringreader inp(s);   \end{verbatim}
or
\begin{verbatim}
   stringreader inp( std::move(s) ); \end{verbatim}
\end{itemize}

\noindent
Once one has chosen the reader of one's fancy, one can call
\begin{verbatim}
   auto p = readandclassify( cl, inp ); \end{verbatim}
The result has type \verb+std::pair<tclass,size_t>+, 
where \verb+tclass+ is the recognized class and
\verb+size_t+ the number of characters read. 
We recommend to use \verb+readandclassify+ as follows: 
\begin{verbatim}
symbol gettoken( reader& inp )
{
   static auto cl = buildclassifier( ); 
      
restart: 
   if( !inp. has(1))
      return a symbol that marks end of file.

   auto p = readandclassify( cl, inp );
   if( p. second == 0 )
   {
      inp. commit(1); // to avoid looping forever
      return a garbage token
   }

   if( p. first is among the whitespace tokens )
   {
      inp. commit( p. second ); 
      goto restart;
   }

   if( p. first is a token that has an attribute )
   {
      attr = the attribute (use inp. get( )). 
      inp. commit( p. second ); 
      return symbol( p. first, attr );
   }

   inp. commit( p. second ); 
   return symbol( p. first ); // without attribute.
} \end{verbatim}  
If choosing is hard, one can also write: 
\begin{verbatim}
template< typename S > symbol gettoken( S& inp )
{
 ...
} \end{verbatim}

\noindent
If the user wants to create a different type of reader, this 
can be easily done. Both \verb+filereader+ and
\verb+stringreader+ have short implementations.

\section{Making it Efficient}
 
The tokenizer in the previous is convenient, but
not efficient at all. The classifier is recomputed every time
the program is restarted, and direct use of the
classifier is not efficient, even when it is deterministic. 
MaphTT provides two solutions:
\begin{enumerate}
\item
   Create a table based classifier, like Lex or JFLEX.
\item
   Translate the classifier into directly executable
   $ C^{++} $ code. 
\end{enumerate} 

\noindent
The interface is primitive:
In the function that builds the classifier, 
one prints code that can be later compiled, and 
which can replace the call to \verb+buildclassifier( )+.

Since classifiers are polymorphic in 
the character set \verb+C+ and the type 
\verb+T+ used for classification,
one needs to provide functions that print
\verb+C+ and \verb+T+.
In case of \verb+C = char+, this is easy, because
\verb+char+ can be printed as integers:
\begin{verbatim}
   []( std::ostream& out, char c ) { out << (int)c; } \end{verbatim}
One can also print \verb+char+ with single quotes, but then
one has to take care of non-printable characters, which makes 
the function too complicated to define in a single 
lambda expression. 

Dealing with \verb+tclass+ is a bit more tricky,
because \verb+enum+ is not well-designed. 
The best solution is to print them by their
names, but then one has to define a print function
for \verb+tclass+, and we didn't do 
that\footnote{the parser generator will do it automatically}. 
For the moment, we hack ourselves out of the situation by
creating code that casts integers into tclass: 
\begin{verbatim}
[]( std::ostream& out, const tclass t ) 
      { out << "((tclass) " << t << ")"; } ); \end{verbatim}

\noindent
We describe the details: 
\begin{itemize}
\item
   If one wants to create a table based classifier,
   one must use 
\begin{verbatim}
template< typename C, typename T > 
void printdeterministic(
   const std::string& Cname, const std::string& Tname,
   const std::string& name,
   const lexing::classifier<C,T> & cl, std::ostream& file,
   const std::function< void( std::ostream& out, const C& c ) > & printC,
   const std::function< void( std::ostream& out, const T& t ) > & printT ) \end{verbatim}
The parameter \verb+Cname+ is the name of the character type,
which \verb+"char"+ in our case. Similarly
\verb+Tname+ must be \verb+"tclass"+. If \verb+C+ or
\verb+T+ is in a namespace, the namespace must be included.
Parameter \verb+name+ is the name that the function will receive.
It should be something like \verb+"builddetclassifier+.

Choose an unused file name (it will be overwritten!), 
and insert
\begin{verbatim}
std::ofstream file( "det.h" );
lexing::printdeterministic< char, tclass > (
   "char", "tclass", "builddetclassifier",
   cl, file,
   []( std::ostream& out, char c ) { out << (int)c; },
   []( std::ostream& out, const tclass t ) 
      { out << "((tclass) " << t << ")"; } );
file. close( ); 
\end{verbatim}
before the return statement. 

When \verb+buildclassifier( )+ is called, it will construct
the deterministic classifier. 

Include the resulting file before function
\verb+gettoken+, make sure that it compiles,
and replace
\begin{verbatim}
symbol gettoken( reader& inp )
{
   static auto cl = buildclassifier( );
\end{verbatim}
by
\begin{verbatim}
symbol gettoken( reader& inp )
{
   // static auto cl = calculator( );
   auto cl = builddetclassifier( ); \end{verbatim} 

\item
   The procedure for creating a classifier
   in $ C^{++} $ code is similar.
   The function that creates the code is:
   \begin{verbatim}
   template< typename C, typename T >
   void printcode(
      const std::string& Cname, const std::string& Tname,
      const std::string& space,
      const lexing::classifier<C,T> & cl, std::ostream& file,
      const std::function< void( std::ostream& out, const C& t ) > & printC,
      const std::function< void( std::ostream& out, const T& t ) > & printT ) \end{verbatim}
The generated function will be called \verb+"readandclassify"+, 
preceded by the namespace \verb+space+.
If \verb+space == "myspace::"+, then the generated
function will be called 
\verb|"myspace::readandclassify"|. 

\noindent
At the end of \verb+buildclassifier+,
insert: 
\begin{verbatim}
std::ofstream file( "directlycoded.h" );
lexing::printcode< char, tclass > ( 
   "char", "tclass", "",
   cl, file,
   []( std::ostream& out, char c ) { out << (int)c; },
   []( std::ostream& out, const tclass t )
      { out << "((tclass) " << t << ")"; } );
file. close( );
\end{verbatim}

Include the resulting file before function
\verb+gettoken+, make sure that it compiles,
and replace
\begin{verbatim}
symbol gettoken( reader& inp )
{
   static auto cl = buildclassifier( );
\end{verbatim}
by
\begin{verbatim}
symbol gettoken( reader& inp )
{
   // static auto cl = calculator( );
   int cl = 22; // Any int will do \end{verbatim}

   Since the directly coded classifier 
   does not need any additional data, 
   one could in principle remove the variable
   \verb+cl+, but if one does that, one
   may have to put it back later when one wants to
   change the classifier again. 
   Because of this, we put an int in place of the classifier. 
   The generated function \verb+readandclassify+ has
   \verb+int+ as first parameter.
\end{itemize}


\end{document}
 
